%%%%%%%%%%%%%%%%%%%%%%%%%%%%%%%%%%%%%%%%%
% Awesome Cover Letter
% XeLaTeX Template
% Version 1.1 (9/1/2016)
%
% This template has been downloaded from:
% http://www.LaTeXTemplates.com
%
% Original authors:
% Claud D. Park (posquit0.bj@gmail.com)
% Lars Richter (mail@ayeks.de)
% With modifications by:
% Vel (vel@latextemplates.com)
%
% License:
% CC BY-NC-SA 3.0 (http://creativecommons.org/licenses/by-nc-sa/3.0/)
%
% Important note:
% This template must be compiled with XeLaTeX, the below lines will ensure this
%!TEX TS-program = xelatex
%!TEX encoding = UTF-8 Unicode
%
%%%%%%%%%%%%%%%%%%%%%%%%%%%%%%%%%%%%%%%%%

%----------------------------------------------------------------------------------------
%	PACKAGES AND OTHER DOCUMENT CONFIGURATIONS
%----------------------------------------------------------------------------------------

\documentclass[12pt, letterpaper]{awesome-cv} % A4 paper size by default, use 'letterpaper' for US letter

\geometry{left=2.54cm, top=2.54cm, right=2.54cm, bottom=2.54cm, footskip=.5cm} % Configure page margins with geometry
 
\fontdir[fonts/] % Specify the location of the included fonts

% Color for highlights
\colorlet{awesome}{awesome-red} % Default colors include: awesome-emerald, awesome-skyblue, awesome-red, awesome-pink, awesome-orange, awesome-nephritis, awesome-concrete, awesome-darknight
%\definecolor{awesome}{HTML}{CA63A8} % Uncomment if you would like to specify your own color

% Colors for text - uncomment and modify
%\definecolor{darktext}{HTML}{414141}
%\definecolor{text}{HTML}{414141}
%\definecolor{graytext}{HTML}{414141}
%\definecolor{lighttext}{HTML}{414141}

\renewcommand{\acvHeaderSocialSep}{\quad\textbar\quad} % If you would like to change the social information separator from a pipe (|) to something else

%----------------------------------------------------------------------------------------
%	PERSONAL INFORMATION
%	Comment any of the lines below if they are not required
%----------------------------------------------------------------------------------------

\name{R. Drake}{Asberry}
\address{804 San Luis Obispo Pl Unit 3, San Diego, CA 92109}
\mobile{Cell:(304) 319-1646, Alt:(520) 338-9388}

\email{drake.asberry@gsa.gov}
%\homepage{www.posquit0.com}
\github{drakeasberry}
\linkedin{www.linkedin.com/in/drakeasberry}
%\skype{skypeid}
%\stackoverflow{SOid}{SOname}
%\twitter{@twit}

%\position{Software Engineer{\enskip\cdotp\enskip}Security Expert} % Your expertise/fields
%\quote{``Make the change that you want to see in the world."} % A quote or statement

%----------------------------------------------------------------------------------------
%	RECIPIENT/POSITION/LETTER INFORMATION
%	All of the below lines must be filled out
%----------------------------------------------------------------------------------------

\newcommand*{\companyaddress}[1]{{\fontsize{11pt}{1em}{\bodyfont\color{black}#1}}} % The company being applied to
\newcommand*{\recipients}[1]{{\fontsize{11pt}{1em}{\bodyfont\color{black}#1}}} % The company being applied to
\letterdate{\today} % The date on the letter, default is the date of compilation

\lettertitle{Specialist, Geographic Information Systems} % The title of the letter

\letteropening{Dear Krissandra McNeill,} % How the letter is opened

\letterclosing{Sincerely,} % How the letter is closed

\letterenclosure[Attached]{Résumé} % Any enclosures with the letter

\makecvfooter{\today}{R. Drake Asberry~~~·~~~Cover Letter}{} % Specify the letter footer with 3 arguments: (<left>, <center>, <right>), leave any of these blank if they are not needed
  
%----------------------------------------------------------------------------------------

\begin{document}

%\makecvheader % Print the header

%\makelettertitle % Print the title
%\vspace{-15mm}
%----------------------------------------------------------------------------------------
%	LETTER CONTENT
%----------------------------------------------------------------------------------------

\begin{cvletter}

%------------------------------------------------

%\lettersection{About Me}
\companyaddress{\emph{21 August 2021} \bigskip \\ \bfseries{National Geospatial-Intelligence Agency\\}}
\vspace{-2mm}

\recipients{Dear Hiring Committee:\\}

\vspace{-2mm}

I am writing in reference to the Training and Education Officer position that is open in the National Geospatial-Intelligence Agency. This position is appealing because it is the culimination of my interests, formal education and professional experiences. 

As a first generation college, I constantly found myself at odds with classmates and professors because my approach to problem solving and critical thinking took on a very different form than what they expected. I attempted to fit their mold and become more like the perceived norm in my new environment. The focus of my bachelor's degree in Landscape Architecture centered around urban planning and land management. While those around me took a very creative and artistic type of design approach in their thinking as well as presentational approach, I found that I took a much more hands-on, practical and analytical approach to design, which influenced the more engineering type presentational approach. 

My undergraduate education laid the foundation for my previous GIS positions. Early on in my career, I was fortunate enough to work with the National Geospatial Development Center on a project that was tasked with digitizing historical soil surveys for online archiving. I was the first hire for the project and was tasked with testing the training material. Following the initial test survey, I naïvely asked why the training was using the specific technology software when our computers were already equipped with more up-to-date software that had the potential speed up the process. This questioning of process placed me quickly on conference call with people who had created the training to which they told my superiors to let try the alternate solution. While I did not have a name for it at the time, this was my first pipeline I created to improve work efficiancy. The completed pipeline was supplemented by collaborative script writing with a student worker and programmer, which allowed us to reduce the average completion time per survey from 16 hours to 4 hours. I was rewarded with being able to write my first training manual, which was I used for training all other student hires. While the initial position I was hired for evolved into a leadership or trainer and quality control duties, it opened up time for my own introduction and training to ArcGIS. Once again my non-traditional approaches coupled with a like-minded colleague were put to use in trying to present solutions to a soil-based construction problem. Rather than creating something completely digital with ArcGIS, we created physical layered project using a basemap and multiple clear mylar plots that representedindividual layers of construction mitigation costs. While the full project showed the current state of soil related construction costs in Indiana, each layer could be peeled back removing one restriction. The peeling back of the layer represented that this factor would need to be mitigated during construction phases, which allowed more developable land to be available. The map updated the legend to match the new visual display of developable land and removed layers that were more cost-effective to the constrcution industry before those that were more cost prohibitive. 

My experience with the National Geospatial Development Center later led to work with a non-profit organization as a AmeriCorps VISTA* volunteer. In this position, I collected water samples, managed watershed maps and databases in an effort to reclaim polluted waterways affected by acid mine drainage. I also taught GIS workshops to other non-profit organizations as well as local, state and federal government agencies. The collaboration in this position with West Virginia University led to my next job offer with the Hydrology Research Center where I continued to work on the project I started as an AmeriCorps VISTA volunteer and gained access to working with larger state and federal databases of water quality data for the region. By this point, I had learned that my outside the box approach to design and work was not a hinderence, but rather an asset that I could captalize as I continued my education. I was able to put this practice in my final year of my undergraduate capstone project where I took pride in the differences I could achieve by digging into the mathematics involved in natural stream design and restoration and presenting these ideas through engineering drawings and topology maps rather than the colorful artistic urban landscape designs.

Following graduation, I began working in the oil and gas industry where I was exposed to a very different use of GIS in the workplace. I found myself learning to map company assets where actively growing databases of information needed to be conflated with historical maps and datasets. The biggest difference here was the effect this information had immediately on day-to-day operations. It was my first experience with development and production servers which were used to provide a versioning framework to protect data integrity and ability to react ever-changing workplace demands. While my main job was to map company assets, my secondary role was to provide additional training to field personel for our high-precision GPS equipment. This became the most salient use of my diverse background of physical work abilities and education. It was my background that helped me be the middleman between those pipeliner who respected for my hard work ethic and ability to do and those upper-level executives who repsected me for my educational achievements. It was in this position that I was able to translate the needs of these two divergent groups to the other. This resulted in the mapping and asset tracking goals of the executives being met while not overburdening the workforce tasked with gathering the data needed to complete the task. It took more than communication though because this was not the first time field staff had been tasked with GPSing field assets, yet they were still using maps from as far back as the 1950s. The first step was to simplify the GPS process based off of their expertise, which led to simplified menus on the GPS unit by excluding all options not used in our company and ordering items in terms of the frequency they were encountered in the field. Once we had trained the influencers of the field staff on GPS equipment and began to provide them with updated maps to work with, we had the buy-in needed for project success. The GPS data began to roll in faster than it had before and the quality of data improved simultaneously. These small improvements in the process of data collection, made integrating the new data much more efficient. This led to more updated maps to field staff and a larger, more accurate database of company assets for the executives. This utlimately led to a larger pipeline where the newly mapped assets were utilized by the ticketing work management system. This issued work orders in real time to field staff and allowed for better tracking and data analysis for exectuives, reports and compliance issues to the multiple groups within the company that needed them.

Realizing that I wanted to have more hands-on experience I left the traditional line of work and started contracting my own business to engineering firms working on new gas pipeline projects. In this role, I led surveying crews of 3-5 men in order to mark and map the path of proposed pipelines. It was here that I gained more in-depth knowledge of collecting quality data. While I was no longer responsible for the mapping of these data points, the approach I took to our field work kept in mind what the mappers would need to have in order to do their job. This mindset led to productive conversations that help efficiency map production and engineering plans while helping us to prioritize of tasks and levels of detail being collected in the field. Leading these survey crews also gave me the opportunity to help develop other employees as well. Rather than taking the traditional stance of crew cheifs who make all decisions as they see fit, I took the approach to involve the crew members in decision making tasks. This not only improved the moral overall, but helped individual crew members understand the roles and responsibilities of other crew members. Many of the crew members that worked under me went on to higher positions in crew ranks and even to future crew cheifs. It was at this point that I realized that while I had technical skills and the ability to learn new ones easily, I needed to learn to be a better communicator.

Since 2013, I have been focusing my career and education on learning to make information accessible to broad audiences. I have studied linguistics and how to teach langauges to help students learn to communicate clearly in a second language. I taught English as a second language to a broad range of international students with diverse cultural background at West Virginia University. My time as an ESL instructor has helped me become a more culturally aware and sensitive person who views life through a global lense. In addition to teaching, I was also in charge of the communication partners program that connected ESL learners with native speakers in the community for the purposes of language exchange and practice. During this time I completed my masters degree in Linguistics and decided to continue this path of education into doctorate level studies. 

I began my PhD studies at the University of Arizona in 2016. It was at this point that I wanted to focus on tying my previously gained technical skills into the my education on second language acquisition. I used this platform to developed strong written and oral communication skills, gained additional cultural competence, and created a positive, productive, learning and working environment. As I taught Spanish as a foreign language, I learned to work with a very different student population that I had previously dealt with in the international students. To my surprise, teaching domestic students who shared cultural values with me was not the easy task I had anticipated. Instead I learned that communication takes skill and dedication to be effective regardless of how much shared or ground is shared between parties. 

My initial idea was to include geospatial components in my dissertation of language acquisition, but I was unable to find buy-in from the professors I had to choose from when forming my committee. As a result, I decided to find other ways to integrate the skills I had learned during my geospatial related jobs. I began to look at finding larger databases of information and think about ways to extropolate patterns that may be present in them. This turned my attention to more cognitive science based aspects of language research. This allowed me to bring previous programming knowledge to my experimental designs. Ultimately, my dissertation was designed and created in python based software and JavaScript for online portions. I excelled in approaching the dissertation from this programmatic approach and began investigating reproducible research methodologies. I was able to then connect with programming research group in Tucson, AZ where I became an integral member of their organization. The organization is an open source of information and people who volunteer their time to help researchers from the academic and local communities overcome technical barriers to their research. I have helped people overcome software installations, dependecy problems and research obsticles in social and behavioral sciences, agriculture and natural resources, engineering, and medical discplines. I was nominated and selected as the Data Science Ambassador for the College of Humanities where I connected researcher with people and programs on campus to support academic research. I have taught at workshops on utilizing reproducible research methods using containerization, cloud computing and documentation. I have also taught various workshops designed for beginning programmers for python, UNIX and git. Of course, while teaching and supporting these folks, I continued to learn many things myself, including how to program in R to run statistical analysis and conduct geospatial analysis with R and QGIS. I learned how to create pipelines where my dissertation designed in python was run, the data wrangling utilized both python and R, and the manuscript was written in LaTeX, which allowed the document to be dynamic and update as new data became available and was analyzed. 

The most valuable part about this experience was the immediate applications of these skills and the development of my communication abilities to talk about these processes to people from technical and non-technical backgrounds. Through the development of this this capacity, I have developed partnerships between different departments  to meet the needs of the universities that will outlast my tenure on location. As my time progressed, I became a coordinator for the Department of Spanish. I found ways to utilize this knowledge of technology and programming skills to make the time of other coordinators on our team more efficient. This allowed us to focus more on the content of the courses we mangaged and invest more into training and development opportunities of instructors who were teaching them. I have generated scripts that create dynamic reports, which are used every semester to update the 26 different syllabi used by the Department of Spanish. I have implemented development sites for our learning management system that are used to push out live production sites every semester. I created training videos of the processes that I developed so that future employees in that role are not starting out blind as I did.

The Training and Education Officer position provides a perfect opportunity for me to continue developing my technical skills and ability to clearly convey complex information to all types of audiences. My corporate work experience has taught me to think analytically and critically in order to resolve issues. My academic experience has taught me how to deliver technical information that helps others overcome their obstacles while maintaining patience and a sense of understanding with those who are seeking assistance. I thank you for considering me as a candidate for this position and I look forward to discussing this opportunity with you.


%------------------------------------------------

\end{cvletter}

%----------------------------------------------------------------------------------------

\makeletterclosing % Print the signature and enclosures

\end{document}